%5. Diskussion. Nennen Sie die Zielsetzung Ihrer Arbeit und fassen Sie 
%zunächst die wichtigsten Ergebnisse in Bezug auf Ihre Hypothesen in einem (oder 
%wenigen) Absatz/Absätzen zusammen. Nennen Sie dann Ihre Schlussfolgerungen 
%(Ihre Interpretation der Ergebnisse) und begründen Sie diese. Zitieren Sie hier auch 
%noch einmal die wichtigen Arbeiten, auf die Sie sich stützen und die helfen, die 
%Ergebnisse zu verstehen. 
%Diskutieren Sie, wenn möglich, auch alternative Erklärungen und legen Sie dar, 
%was für Ihre Interpretation der Ergebnisse spricht. Gehen Sie auch ausführlich auf die 
%inhaltlichen und methodischen Grenzen Ihrer Untersuchung ein und beschreiben Sie, 
%welche Anschlussfragestellungen und weiteren Forschungsbedarfe sich aus Ihrer 
%Studie ergeben. Reflektieren Sie abschließend auch theoretische und praktische 
%Implikationen Ihrer Arbeit. 


% Aus dem Bewertungsschema:
% Ergebnisse werden im Hinblick auf Hypothesen und Fragestellung angemessen zusammengefasst. 
% Es werden alle Untersuchungsfragen beantwortet. 
% Die Ergebnisse werden aus theoretischer Perspektive interpretiert.  
% Es wird deutlich aufgeführt, was für die eigene Interpretation der Befunde spricht.
% Alternative Interpretationsmöglichkeiten werden diskutiert. 
% Die Ergebnisse werden aus methodischer Perspektive interpretiert. 
% Potentielle Einschränkungen der Untersuchung und des eigenen Vorgehens werden benannt und ihre Konsequenzen für die % Interpretation diskutiert. 

% Theoretische und praktische Implikationen werden abgeleitet. 
% Die praktische Relevanz wird verdeutlicht. 
% Weiterer Forschungsbedarf und Ideen für zukünftige Forschung werden beschrieben. 

%\section{\thesection. Section Title}
%\subsection{\thesubsection. Subsection title}
%\subsubsection{\thesubsubsection. Subsubsection title}
%\paragraph{\theparagraph. Paragraph title}%\section{\thesection. Section Title}
%\subsection{\thesubsection. Subsection title}
%\subsubsection{\thesubsubsection. Subsubsection title}
%\paragraph{\theparagraph. Paragraph title}

\section{Discussion}
The aim of the present was twofold. 
First, to provide a psychometric analysis of the ICU from a network perspective and compare and contrast the results obtained to existing research.
Second, to apply network analyses techniques to extend current knowledge regarding the role of the LPE specifiers with respect to 
various form of anti social behavior. ICU correlates (intimate partner violence, ....)
The second aim could unfortunately not be implemented as intended due to insufficient variability in the dataset at hand.
The following section will hence discuss the results from the psychometric network analysis.

\subsection{Influence of modeling decisions - Multi-verse approach}
The point estimates obtained implementing the various modeling decisions varied considerably.
However, the approach of this study was exploratory in nature (as opposed to predictive or explanatory). 
Consequently, my interpretation will not rely too heavily on point estimates.
Combining the various point estimates as well as boot strapped confidence interval allows to distill the results into general tendencies.
These seem to be quite robust across individual modeling choices and will hence be the focus of result interpretation.  

\subsection{Psychometric network of ICU items}
Regarding the first aim, several findings from the psychometric literature could be replicated.
Method factors could also e found with the spinglas algorithm detecting a community of nodes that represented all positively worded items.
Some evidence for content factors was also found with one community consisting entirely of nodes that correspond to items from the "careless" subscale.
The comparably high missingness of item \#19 ("I am very expressive and emotional (-)") which, with 10.39\% missing cases, was on average more than five times as often as the other items suggests 
missingness that is not merely attributable to occasional sloppy answering style.
A closer look at the German translation (\textit{"Ich bin sehr extrovertiert und gefühlsbetont."})[sic!] suggest that the missingness might be attributable to
students not being familiar with the term "extrovertiert" (which should actually be "extravertiert", a common mistake that 9th grade students are likely unaware of).
Should this be the case, missingness for this item might not be at random as it might be related to factors such as verbal intelligence or parental education.

\subsubsection{Network edges}
Most estimated networks were rather dense.
This is likely due to the fact that the scale consists of a large number of highly similar items.
A closer look at the network edges suggests that a large number of strong edges are likely due to similarity in item wording. 
In the German translation the item wordings of some items are more alike compared to the English original.
Items \#1, \#6 and \#22 have very similar item wording and are arguably semantically closer compared to the English original.
Both feelings and emotions are translated with the German word "Gefühle". 
It might be argued, that adolescents might not be aware of the subtle differences of these two concepts, yet the primary 
... of the German items suggests the physical expression of emotionality is referred to while the original English items could also cover the verbal expression of inner states.
Similarly the item pair \#8 and \#21 are arguably semantically closer in the German translation that in the English original. 
"Concerned" and "unimportant" are translated with "wichtig" and "unwichtig" resulting in the two items almost reading as exact opposites of one another.
The strong relationship of item \#16 and \#17 might be a combination of similar wording (both contain "hurt"), carryover due to their location as well as both od them being positively worded.
The combination of these three features might explain the strong relationship. While \#16 is intended to tap into the callous domain it might have captured more of the uncaring domain in this context.
This might also have been enhanced by the German translation reading "Wenn ich jemanden verletzt habe, entschuldige ich mich" vs the English original "I apologize ("say I am sorry") to persons I hurt."
The focus of the German translation is arguably shifted away from the callous habit of (not) apologizing towards an reaction to having hurt someone (i.e. an act of caring).
The opposite, i.e. more semantic distance to other items of the domain can be observed for item \#12. The English "uncaring" has been translated with "gleichgültig" (indifferent). 
In combination with "kalt" (cold) item \#12 reads as if it refers to emotional expressions rather than behavior which might explain its closeness to items from the unemotional domain in the network.

Most edges are positive (as would be expected) however, some negative edges can be observed.
While most of these negative edges are very small and can likely be ignored two patterns seem to emerge.
First, the highest number of negative edges are connected to nodes belonging to the "unemotional" dimension.
The unemotional dimension of the ICU has received considerable criticism with the notion of the Unemotional items potentially tapping into different disorders \parencite{cardinale_reliability_2017}. This could explain the observed negative edges.
Furthermore the items \#2, \#10, and \#13 which have previously been criticized for their poor psychometric properties have a high number of negative edges providing further evidence of them not measuring the intended construct. 

\subsubsection{Node centrality}
The amount of negative edges is also depicted when comparing the node strength to the node expected influence. 
While the former represents the sum of absolute edges for a given node, the latter takes the edge sign into account and subtracts negative edges. 
Nodes for which node strength and node expected influence differ considerably are those with more or stronger negative edges.
As stated in the previous paragraph, nodes corresponding to items on the "unemotional" subscales are associated with a large amount of negative edges, with one noteworthy exception.
Item \#6 "I do not show my emotions to others" is among the most influential in the network.
Node centrality can be interpreted in various ways, from a more traditional latent variable perspective it is comparable to factor loadings. 
Indeed the most central nodes in the present analysis do correspond to those with the highest factor loadings presented in a recent paper based on the same dataset
\parencite{kliem_factor_2019}.
From a network theory perspective central nodes exert a greater influence on the entire system. 
These nodes might stabilize and "communicate" through their high number and or strong connections to other nodes.
This logic might be exemplified by node \#21 "The feelings of others are unimportant to me", 
not endorsing this will consequently lead to being concerned about other peoples feelings \#8,
and hence doing things to make them feel good \#24,
and not trying to hurt other's feelings \#17 which in turn will lead to apologizing should one have hurt someone \#16.
Item \#16 is a nice example of an item potentially undesirable in a latent variable context. 
An apology can be an expression of guilt and/or an expression of care. 
It can hence be expected to exhibit cross-loadings on the uncaring subscale \parencite{essau_callous-unemotional_2006}.
While cross-loadings are a  nuisance and threaten the "purity" of the latent construct they are potentially valuable from a network perspective as they potentially explain how the different symptoms are related
and information might "spread" through the network. 
Such nodes might furthermore prove useful as potential targets for interventions.

This idea is developed further in the centrality indices on a meso-level.
On the community level nodes with a high number of edges within the cluster are interpreted as stabilizing this facet whereas nodes with a high number of connections to other communities are termed
bridges \parencite{robinaugh_identifying_2016} or communicating symptoms \parencite{blanken_role_2018}.

From visual edge inspection it can be seen that the "unemotional" dimension as well as the "uncaring" dimension have a high number of strong within factor edges. This can have two possible interpretation. From a network perspective this would be interpreted as these indicators closely interacting and reinforcing each other. While this might partially be the case the second explanation, i.e. a high topological overlap due to very similar item wordings seems to be another possible explanation. In light of the debate surrounding the factor structure this is positive as \textcites{kliem_factor_2019} criticised the factor structure suggested by \textcites{essau_callous-unemotional_2006}
to be inconsistent with regards to item content.

These interpretation show that the theoretically assumed factor structure might by justified but very closely tied to fine semantic nuances that are not sustainable in translation. 
They might further be influenced by item order and coding.
 
 
\subsubsection{Dimensionality}
Two dominant communality patterns emerged from community detection with the walktrap algorithm.
The community structure that was uncovered most frequently was a 2-community solution with all positively worded items in one community and all negatively worded items in a second.
This is in line with multiple previous findings who discuss a strong influence of item wording on the factor structure of the ICU.

%\subsection{Psychological network analysis}
%
%\subsubsection{Network edges}
%Due to the low prevalence of the anti-social behavior items a large number of boot strapped samples resulted in non-positive definite correlation matrices.
%This was likely due to the fact that some of the bootstrap samples might not have contained any participants exhibiting the behavior in question.
%
%Whereas the psychometric networks containing only the ICU items were very stable across the estimation methods. This is not the case for the psychological networks.
%There were two major network structures emerging. One with the ICU factors where isolated from the behavioral items and a second where they were connected.
%
%A lack of connection of the CU-trait nodes with the indicators of anti-social behavior indicates that there was no unique variance between any these of these nodes.
%There are several potential explanations for this observation. 
%One is a general lack of variability - what does not vary also cannot co-vary. 
%A second possibility is a great topological overlap of the behavioral items and/or the ICU facets.
%The string connections within the behavioral items as well as within the ICU facets suggest this interpretation. 
%In addition the common method variance of the ICU facets and the behavior items correspondingly might account for a large share of variance. 
%
%However, within both the unconnected models as well as the connected solutions the estimated edges were very consistent. 
%For both the unconnected as well as the connected networks it was evident that the "unemotional" subscale was the most isolated one. 
%Its only week connection to the rest of the ICU subscales was consistently the "uncaring" factor. 
%This is in line with previous research indicating that the "unemotional" subscale shows the lowest correlation wit th ICU as a whole as well as with external criteria that the ICU is generally predictive of \parencite{cardinale_reliability_2017}
%In the connected network the connection between the ICU subscales and the behavioral items is consistently via the "callousness". 
%The strongest edge thereby being the "callousness" subscale and the extortion item.
%This as well is in line with previous research indicating that the callousness subscale shows the strongest associations with anti-social behavior.
%
%Accounting for the anti-social behavior the ICU facets did not constitute a latent factor in the sense that the network comprising the four CU-trait nodes was not fully connected.
%The unemotional and the careless facet where not connected in any of the estimations of the multi-verse. 
%The edge of the unemotional and the callousness facet was consistently weak. 
%The other three facets (uncaring, careless and callousness) consistently formed a cluster across all estimations. 
%
%These results are problematic with regard to the LPE specifier requiring two of the four facets to be present.
%The present analysis suggests that those facets are not equivalent, both in their internal organization as well as their relationship to behavioral covariates.
%While the uncaring facet is central with regards to within construct relationships and hence potentially important for the stability of the symptom network, 
%it is the callousness facet that showed unique associations with the anti-social behavior nodes. 
%
%
%Another noteworthy consistency among the estimated networks is that with the exception of the extortion item all other edges of the "callousness" facet with the behavioral items are negative.
%There are three possible explanations for this observation.
%(1) This could be indicative of a suppressor effect.
%(2) All networks which show the negative edges are based on polychoric correlations. 
%Polychoric correlations are known to cause spurious negative correlations in case there are too few observations in cross-tables. 
%Given the low prevalence of some of the items this is certainly a possibility in the data at had. 
%Due to this known weakness it is recommended to always compare networks based on polychoric correlations to their counterpart based on Spearman correlations \parencite{epskamp_tutorial_2018}.
%(3) 
%



%The high number of negative edges in the estimated networks is very evident. 
%\subsubsection{Node indices}


\subsection{Limitations}
The present study suffers from a several important limitations.
In this paragraph I will discuss these limitations pertaining to measurement concerns,
the sample population, methodological issues as well as theoretical shortcomings 

\subsubsection{Measurement related limitations}
First, the data only stems from self-report. 
Additional versions of the ICU are available (parent report and teacher report).
Findings should hence be compared to those obtained by other report. 
Furthermore, the ICU comprised question 873 to question 897 of a 901 item long survey form.
This location at the end of the questionnaire might have caused additional measurement error and facilitated the emergence of method artifacts in the form of common method variance.
The comparably high amount of missingness (see \ref{fig:missingness}) supports this possibility.
%Common method variance could be responsible could be responsible for some observed edges. 
However, most of common method variance (unless unique to a pair of items) should be excluded by the use of partial correlations.

The present study is a secondary analysis of existing data. 
The data hence were not collected with the present study in mind. 
Hence the questionnaires analyzed in the present study might not ideally capture the intended constructs.
Future studies employing network methodology should aim for measures that are tailored for use in a network analysis as well as in combination with the ICU.

The items assessing anti-social behavior were all school based. 
This had the obvious advantage that their wording was close to the everyday experiences of the participants and these items hence showed reasonable prevalence for the present analysis.
An obvious drawback is, that showing some of the more extreme behaviors (such as hitting a teacher) makes it likely to get expelled from school and hence not participate in the survey.
It can hence be expected that there might me a slight selection bias with respect to the top end of the scale.


\subsubsection{Limitations regarding Sample Population and Study Design}

The age range of the present sample was rather narrow due to the participants all attending the same grade.
While this provides a potentially more complete picture of that very age group it might limit generalizability of results.
The narrow age range might have provided additional challenges given the general low variability of the data at hand.

The data analyzed in the present study is cross-sectional, hence no causal conclusions are possible. 
While network theory highly relies on processes and interacting systems, the analysis of cross-sectional data for the generation of hypotheses is common. 
The exploratory insights obtained in the present study will nevertheless have to be replicated and confirmed in longitudinal designs.
It has to be furthermore reiterated, that the observed between-person effects should not be confused with within-person effects.

The assessment of pathology in a general population sample is a common practice. 
Conclusions that can be inferred about the population with high scores on the trait of interest, however, are limited.
Yet, the opposite also holds. Conclusions from a population characterized by features that are highly related to the trait under investigation likely suffer from collider bias. 
Future studies should ideally focus on a selection of participants that is independent of the variables in the model, yet at the same exhibit enough variability on the constructs of interest. 
In the context if network models careful phrasing of distinct items, that are meaningful yet similarly interpreted for a broad range of populations could help in this endeavor.
A focus solely on children who already have CD could limit advances in our understanding of how CD develops (Frick et al. 2014) \parencite{frick_can_2014}

\subsubsection{Methodological Limitations}
 
 Network analysis is a rather recent technique. 
 Hence, methodology is still under development and best-practice recommendations regularly changing.
 The R package \texttt{bootnet} which was used for most of the present analyses for example was updated xz times
 during the preparation of the manuscript at hand. 
 While utmost care was taken to carry out the analyses in accordance with current recommendations these 
 recommendations might already be obsolete. Several critical methodological work that the present work builds 
 on was still in preprint status. While peer review arguably does not eliminate error these studies which have not undergone peer review might contain errors.
 
 The present study furthermore implemented a multi-verse analysis. 
 While the goal of a multi-verse analysis is to capture the consequences of different yet justifiable analytical decisions the present analysis does not claim to implement the entire possible multi-verse.
 Decision factors that were not implemented were: Treatment of missing data or other estimation approaches (e.g Bayesian \parencite{williams_bggm_2019}).
 Bayesian estimation approaches were not implemented for two reasons: (1) the corresponding methodology is even more recent and less stably implemented than the rest of the methods applied in this thesis, 
 (2) Bayesian approaches are extremely computationally expensive. The current approach already, took several days on a personal computer. 
 The implementation of Bayesian approaches could hence not be feasibly implemented in a multi-verse analysis without the use of high performance clusters.
 
 Another potentially reasonable yet methodologically different approach, would be the dichotomization of data and the corresponding estimation of Ising models \parencite{epskamp_tutorial_2018}.
 Given the highly skewed data in the present study that would have been a viable option.
 However, such a transformation comes with a large number of theoretical and methodological decision such as where to set the cut-off point.
 Furthermore, as the primary goal was the investigation of the ICU from a psychometric perspective, dichotomization would have made it difficult to compare the results to the existing literature. 
 
 
\subsubsection{Theoretical limitations}
 
 One of the most obvious shortcomings is certainly that conduct disorder was not assessed.
 As CU-traits are commonly used to subtype CD using them in isolation might seem theoretically unfounded.
 However, there are theoretical as well as empirical reasons to justify an investigation of CU-traits independent of CD.
 Based on the original conceptualization of Cleckley the \textit{mask of sanity} might allow CU-only youths to remain under the radar of law enforcement
 and mental health diagnoses. 
 It can furthermore be expected that the answers obtained by self-report are a lot less truthful in patient ot forensic settings where participants might suspect negative consequences based on their answers,
 even more so in a construct strongly related to lying and deception.
 A community-based survey as the one at hand might hence provide a more complete understanding of the construct
 compared to clinical or forensic samples. Findings by Andershed and colleagues in a community sample of adolescents furthermore found 
 a large group of adolescents to exhibit only CU-traits without co-occurring CD \parencite{andershed_callous-unemotional_2018}.
 Collins \parencite{colins_clinical_2016} has used the LPE specifier in a non DSM-5 CD centered manner and found..\todo{add findings}
The common practice of identifying a risk group within a community-sample (e.g. by using one or several of the behavioral abundantly present in this Criminological study)
was not considered. Such a practice (although frequently seen) is likely to introduce collider bias and invalidate the findings (see \nptextcite{de_ron_psychological_2019} for an overview in the special context of network analyses).
% Furthermore, it can be argued that the second analysis, including several measures of antisocial
% behavior likely accounted for a large percentage of variance represented by conduct disorder therefor complementing the CU-traits only analysis in that regard.
Nevertheless, inclusion of CD as a control variable in addition to other constructs that have been found to be related or who 
unintendedly obtain increased CU scores such as ODD, ADHD, ASD or various mood disorders might be warranted to improve the specificity of the item wording. 

Future studies, should hence include these constructs ideally with few and carefully chosen indicators (not clinical diagnoses) to aid in the revision of the ICU item wording.

The current analysis, further,  did not include any moderating variables. 
The necessary methodology to accomplish this is available though still being developed \parencite{haslbeck_moderated_2019}.
Furthermore, existing problems with low prevalence and differential item variability would be amplified even further in a moderated analysis.  
Especially gender is repeatedly mentioned as a moderating variable of influence in the assessment of CU-traits \parencite{cardinale_reliability_2017} and should be taken into account in future analyses in more suitable samples.
     

 
 
 \subsection{Conclusions}
 Results of the present psychometric network analysis are in line with the current body of literature on the psychometric properties of the ICU in the general population.
 In addition to known issues (influence of item wording, low external validity of the unemotional subscale) I suggest additional problems with the German translation as well as potential effects of item order.
 
 Generalizability of these findings is likely compromised given the large number of limitations of the present analysis.
 Given the ability of the present study to reproduce a large number of findings from the literature albeit these limitations attest to the stability of these issues.  
 
 The good news seems to be that, from a network perspective, the original APSD items the ICU was based on seem to be indeed central to the symptoms represented by the whole scale.   
 
 The focus on CU traits for early signs of psychopathy with the hope to designate an important yet small subgroup of what might one day become severe, violent and chronic offenders cannot be corroborated by severe measurement issues. 
 A low noise to signal ratio ratio makes a fine tuned assessment tool a necessity. 
 The agreement of these findings with what the literature has consistently reproduced for over a decade has to be translated into improved questionnaire design. 
 Especially in the assessments of traits that have potentially high costs attached to both false positives as well as false negatives. If a bad scale were to be used for the identification of a risk group this would in the best case waste money on interventions for those not at risk and in the worst case stigmatize and harshly criminalize youth and leave those actually at risk undetected. Such measurement issues are of course not unique to the area of psychopathy. In fact issues like item overlap in psychopathology the interpretation of items by participants and the lack of theory in questionnaire construction are classic yet very current. Their increased discussion led some to believe the replication crisis in psychology will be followed by a measurement crisis. 
 

  \subsection{Implications for theory and practice}
 The strong suspected influence that word choice, item order and coding might play in the factor association should be further assessed in future studies and taken into account when the scale is used in practice.
 Measurement invariance tests comparing the factor structure of translations in a homogeneous population (ideally randomized bilingual individuals) might shed further light on the issue.
 Similarly, measurement invariance should be assessed in a sample that has been presented the scale in fixed vs. random item oder to further evaluate the influence of item order.
 Regarding the common critique of the "unemotional" facet  future studies should explicitly assess ASD as well as mood disorders to quantify the amount of variance attributable to internalizing disorders.
 As far as the clinical practice is concerned those diagnoses are likely thoroughly probed for differential diagnosis anyways. 
 Nevertheless, it has to be kept in mind that the wording of the ICU focuses on the suppression of emotional expression and hence likely underestimates other aspects of unemotionality potentially even more relevant in the context of psychopathy.
 
 With regards to the assessment of the LPE specifier the present findings suggest that the four facets are neither equivalent regarding the internal structure nor their relationship to behavioral covariates. Findings obtained with the three dimensional subscale structure translate to the subscale structure more closely resembling the four LPE facets.
 The present findings suggest that this is partially due to suboptimal item wording yet also attest specific influence to specific core items that seem to be vital for the construct as a whole.
 
 Disentangling the effects of item coding, item order nuances of item wording and translation in addition to overcoming general shortcomings regarding construct validity of the current item pool should hence be the first priority of future research including the ICU. 
 Thereafter additional network analyses might help investigating whether the presence of two LPE specifiers is a useful criterion for the subclassifications of CD or whether the interaction of symptoms from different facets might better explain the emergence of specific antisocial behavior, as well as its intensity and perseverance.   
 
 Balancing act between wordings that will not prompt participants to lie, are accessible to self-reflection even in individuals who might have a deficient in that regard while at the same time cover the construct in its entirety.
 
 Future studies, specifically designed for the study of the ICU in non-referred samples might be advised to run extensive simulation studies based on a large body of literature available to fine tune the inclusion of specific behavioral items
 with carefully chosen item wording and corresponding response scale in order to maximize (yet not inflate) the endorsement rate, variability and theoretically expected association with the concept of CU-traits.
 
%\texttt{\textsc{Lahey \parencite{lahey_what_2014} warns about the danger of circularity with the detection of potential subtypes leading to an adaptation of the measurement practices which in turn will confirm the existence of these subtypes. This logic can also be reversed. The failure to detect expected relationships due to shortcomings in the questionnaire might equally effect the revision of theories and assessment practices. Both of these tendencies have to be avoided by the improvement of questionnaires based on the existing knowledge. }}