%1. Einleitung. Die Einleitung umfasst eine kurze Einführung in den 
%Themenbereich und in die Ziele Ihrer Arbeit. 
%Dazu gehört das Erkenntnisinteresse und Begründung der Fragestellung, die Beziehung zu übergeordneten Themen 
%bzw. die Abgrenzung von ähnlichen Themen und am Ende ein Überblick über die 
%nachfolgenden Kapitel. Ggf. kann in der Einleitung zusätzlich eine kurze 
%psychologiegeschichtliche Einordnung erfolgen. Nutzen Sie die Einleitung, um die Ziele 
%Ihrer Arbeit zu verdeutlichen und um Ihre Leser auf die Arbeit neugierig zu machen. 

\section{Introduction}

The idea to identify future criminals before they commit a crime has fascinated humanity for centuries.
While Hollywood might rely on the psychic abilities (e.g. in Minority Report) or character judgments of wise individual (e.g. Star Wars) psychologists and criminologists rely on the identification of specific risk groups.

These groups can then be target with specific interventions too ideally alter the trajectory in a way that averts perpetration.

The factors that define a group at risk can be manifold.
Risk factors can be biomarkers (e.g. specific genes), sociodemographic features (e.g. poverty or low parental education), personal history
(e.g. abuse or trauma) as well as personality traits.

The latter are the focus of the present thesis. \gls{cu} which describe the consistent tendency to lack empathy and concern for others as well as a general lack of pro-social emotions.

This personality trait, which corresponds to the affective dimension of psychopathy, has been repeatedly found to be associated with a 
particular group of perpetrators whose offenses can be characterized as especially severe, violent and persistent.

These findings have been so consistent that CU-traits have been added to the \gls{dsm} in the context of \gls{cd} diagnosis \parencite{DSM-5}.
When children or adolescents are being diagnosed with CD the professional making the diagnostic has the option to add the specifier
\gls{lpe} to the diagnosis when they assess the presence of a sufficiently high level of CU-traits in addition to the diagnostic criteria
of conduct disorder.
Being coded with the LPE specifier designates a specific subgroup of individuals that, in addition to the higher risk of especially severe perpetration,  have for example been shown to be more difficult to treat \parencite{hawes_callous-unemotional_2014} compared to individuals who did not qualify for the classifier.

Adaptations of diagnostic criteria naturally spark the scientific interest in a construct.
While CU-traits are not a new discovery their inclusion in the DSM-5 certainly comes with 
increased responsibility regarding its reliable and valid assessment.

The most frequently used tool for the assessment of CU-traits is the \gls{icu}.
Despite its popularity several issues have been identified that potentially severalty impact its quality.  
Focus of the present thesis will hence be the investigation of the ICU.
This will be done based on data from a large representative study carried out by the \gls{kfn} \parencite{bergmann_jugendliche_2017}.
In addition I will apply a novel technique, i.e. Network analysis to provide a fresh perspective on results obtained with more traditional methods. 

Network analysis attempts to understand a construct (such as a personality trait) by focusing on the unique relationships between the indicators. 
It views indicators (e.g. items) as causal agents that interact with each other \parencite{cramer_comorbidity_2010}. 
This is a fundamentally different perspective compared to the idea of latent variables where the items are a mere reflection of the underlying trait they attempt to capture. 
Such a different perspective might offer a complementary view with potentially new insights.

Section~2 addresses the theoretical as well as methodological background. 
I provide an overview of callous-unemotional traits as a construct, literature findings related to the research question as well as its measurement with the ICU. In addition I will provide more background information pertaining to network methodology.
Section~3 will elaborate on the implementation of the survey from which the data from the present analysis stem.
Methodological details about the implementation of the statistical analysis are also covered in section~3 and the results of the same presented in section~4.
In section~5 I discuss these results in light of earlier empirical findings. 
Furthermore, the limitations of the present study are critically appraised and the findings reflected in light of the existing body of literature. 
I moreover derive recommendation for future research and practice.