%Theorieteil
%2. Theorie.
%Stand der Forschung. Dieser Abschnitt der Arbeit bettet Ihr Thema in die
%Forschung ein und führt auf Ihre Fragestellung(en) hin.
%Berichten Sie aus der Literatur
%die relevanten Begriffsdefinitionen, für das Thema wichtige Theorien und Modelle und
%relevante Forschungsergebnisse unter Beachtung des methodischen Vorgehens.

%Gehen Sie auf besonders relevante Primärstudien ausführlicher ein.
%Benennen und  diskutieren Sie widersprüchliche Annahmen und/oder Befunde. Nutzen Sie diese
%theoretische Einführung also auch dazu, die Grundlagen für Ihre speziellen Fragen
%darzustellen.
%Nutzen Sie Überschriften, um den Theorieteil Ihrer Arbeit sinnvoll zu
%gliedern. Achten Sie hier und in den weiteren Abschnitten Ihrer Arbeit unbedingt
%darauf, alle verwendeten Quellen zu kennzeichnen. Informationen, die Sie Werken
%fremder Autoren wortwörtlich oder dem Sinn nach übernommen haben, müssen Sie mit
%einem Hinweis auf die Quelle kennzeichnen, ansonsten handelt es sich um ein Plagiat.
%Wörtliche Zitate setzen Sie in Anführungszeichen, nennen die Autoren sowie die
%exakte Fundstelle des Zitats, z. B. (Parker, 2011, S. 123). Wörtliche Zitate aus
%englischen Originaltexten werden auch in deutschsprachigen Arbeiten in der
%Originalsprache zitiert. Wörtliche Zitate sollten Sie bitte eher sparsam verwenden.
%Häufiger sind indirekte Zitate, in denen Sie Inhalte mit eigenen Worten darstellen (z. B.:
%Wie Kent und Wayne (2011) anmerkten...). Am Ende eines jeden Abschnitts sollten Sie
%die für die vorliegende Arbeit wichtigsten Erkenntnisse zusammenfassen und in ihrer
%Bedeutung für Ihre Arbeit kommentieren.
%Fragestellung und Hypothesen. Auf Basis widersprüchlicher Befunde oder
%ungeklärter Forschungsfragen formulieren Sie nun Ihre Fragestellung – welche
%Aspekte interessieren Sie in Ihrer Arbeit besonders und was möchten Sie
%herausfinden? Begründen und präzisieren Sie Ihre Teilfragestellungen und
%Hypothesen. Schreiben Sie hier nur Fragen bzw. Hypothesen auf, die Sie später
%anhand Ihrer Daten untersuchen. Formulieren Sie zu Ihren Teilfragestellungen
%konkrete Erwartungen/Hypothesen.
%Die Herleitung/Begründung der Hypothesen muss
%gut nachvollziehbar und belegt sein. Falls es unterschiedliche theoretische
%Erwartungen gibt, können Sie auch konkurrierende Hypothesen formulieren. Nennen
%Sie nur Ihre Hypothesen und nicht die entsprechenden (statistischen) Null- oder
%Alternativhypothesen. Falls nicht zu allen Teilfragestellungen aus der Theorie und aus
%vorhergehenden Befunden Erwartungen/Hypothesen ableitbar sein sollten, können Sie
%diese Teilfragestellung auch als explorativ kennzeichnen.

% Aus dem Bewertungsschema:
% Theorieteil ist klar und sinnvoll strukturiert
% alle zentralen Begriffe werden eingefuert: Definitionen entsprechen aktuellem Forschungsstand
% einschlaegige psycholigische Grundlagen werden mit Verweis auf die Primaerliteratur sachlich richtig dargestellt
% % Aus der Diskussion der Theorien und Modelle bzw. widerspruechlicher Befunde werden Leitfragen/ Hypothesen fuer die Arbeit stringent und kohaerent abgeleitet



\section{Theoretical Background}
\subsection{Callous Unemotional Traits}
Common definitions of psychopathy in adults encompass both affective dysfunctions as well as antisocial behavior.
The affective dysfunction facet is characterized by reduced empathy and guilt as well as diminished attachment to significant others.
When observed in children these characteristics have been referred to as "callous unemotional" (CU) traits, "psychopathic traits" or in the Diagnostic and Statistical Manual of Mental Disorders (DSM-5)
"limited prosocial emotions" \parencite{viding_callousunemotional_2018}

They were first introduced by Hervey Cleckley in 1941 who viewed this affective dysfunction as the hallmark of psychopathy.

Various findings indicated that the presence of the affective features of psychopathy possibly characterizing more severe,
chronic and violent patterns of anti social behavior in adults \parencite{leistico_large-scale_2008} furthermore distinct neurological, cognitive and emotional characteristics \parencite{blair_neurobiology_2013, frick_psychopathy_2018} were identified.
This led researchers to focus on CU traits for the investigation of the etiology of adult psychopathy \parencite{frick_evaluating_2015}.

Similar findings emerged in children and adults indicating that CU traits might characterize a distinct subgroup of antisocial children and adolescents
who differ in from other antisocial youth in terms of severity \todo{continue argument}
Several studies have found CU traits to be uniquely associated with poor treatment outcomes of child and adolescent conduct problems (see \cite{hawes_callous-unemotional_2014} for an overview).
CU traits have been found to be relatively stable during the course of childhood, adolescents and early adulthood \parencite{obradovic_measuring_2007, loney_adolescent_2007}.

This ability to potentially differentiate the most severely antisocial group of youth has led to the inclusion of the specifier "limited prosocial emotions" in the DSM-5 \parencite{DSM-5} definition of conduct disorder.
This specifier can be applied to children diagnosed with conduct disorder if they exhibit at least two of the following four criteria persistently over at least 12 months and in more than one setting or relationship:
(1) Lack of remorse or guilt,
(2) Callous - lack of empathy,
(3) Unconcerned about performance,
(4) Shallow or deficient affect \parencite{DSM-5}.

Naturally, this recent inclusion of CU-traits in the DSM-5 has further increased the research interest in this set of features over other symptom dimensions of psychopathy (i.e. narcissism/interpersonal and impulsivity/lifestyle features). 
A development which has not been uncriticized \parencite{lahey_what_2014}.
Andershed and colleagues found the inclusion of all three psychopathic trait dimensions to be more predictive of various future antisocial outcomes in a community sample of adolescents and hence warn about the sole focus on CU traits \parencite*{andershed_callous-unemotional_2018}.

\subsubsection{Assessment of CU Traits}
CU traits can be assessed with various existing measures in youths and adults.
Traits can be assessed using informant as well as self-report rating on a questionnaire
or involving clinical judgment in structured interview-based clinical rating scales.
The most widely used clinical rating scale for adults is the \gls{pclr}
\parencite{hare_pcl-r_2018} \todo{include proper reference}.
Several measures for application in youths were derived of the latter.
As a psychopathy dimension, CU-traits are frequently assessed within more broad measures of psychopathic traits,
yet several instruments were created to specifically index CU traits.

\paragraph{Psychopathy Checklist - Youth version}

In the \gls{pclr} by \ctextite{forth_psychopathy_1990}
CU features are assessed using four items: Lack of remorse/guilt (\#6),
shallow affect (\#7), callousness/lack of empathy (\#8) as well as failure to accept responsibility for ones actions (\#16).
Unlike other items of the scale which have been adapted from the adult scale to be more applicable to youths,
these affective items comprising the CU dimension however are not modified. 
This might indicate that they are more generalizable across age groups than other symptoms of psychopathy \parencite{viding_callousunemotional_2018}.

\paragraph{Youth Psychopathic Traits Inventory}
The \gls{pclr} more recently developed by \textcites{andershed_psychopathic_2012} is a 50-item self-report scale explicitly designed for use in non-referred youths.
The three facets "callousness" (5 items), "unemotional", (5 items) as well as "remorselessness" (5 items) assess CU-traits and can be combined to a CU factor.

\paragraph{Antisocial Process screening device}
The \gls{apsd} developed by \textcites{frick_antisocial_2001} is a
20-item rating scale designed to measure the psychopathy dimensions CU traits, narcissism
as well as impulsivity in children between the ages of 6–13 years. The APSD was modeled after the PCL-R
but explicitly designed as an instrument for non-referred samples of youths  \parencite{andershed_psychopathic_2012}.
It contains the same number of similar items capturing analogous behavior adapted to children and scored on a similar 3-point scale \textit{not at all true} (0), 
\textit{sometimes true} (1) and \textit{definitely true} (3). 
CU factor includes items regarding manipulative charm, deficits in concern about others, lack of remorse or guilt, insufficient concern about school work, not keeping promises, not keeping friends, and not showing emotions
The CU subscale of the APSD, consisting of six items, has been frequently used as a operationalization of the CU construct in research \parencite{viding_callousunemotional_2018}.

\paragraph{Inventory of Callous-Unemotional traits}
A measure exclusive for the assessment of CU-traits is the Inventory of Callous-Unemotional traits \parencite{frick_icu_2004}. 
It comprises 24 items rated on a 4-point Likert scale ranging from \textit{not at all true} (0) \textit{definitely true} (3). 
The inventory is available in five different versions: other-rated for school-aged children (parent version, teacher version), 
other-rated for preschool children (parent- and teacher-rating version) as well as a self-report version suitable for school-aged children, adolescents and adults.
It was creating by extending four items from the CU subscale of the APSD \parencite{kimonis_using_2015}.
Being the only dedicated measure of CU traits, it and has since become the most influential tool for the assessment of CU traits
\parencite{cardinale_reliability_2017}.

\subsection{Psychometric Properties of the ICU}
Since its conception the ICU has been used in a large number of studies.
Consequently it's psychometric properties have frequently been assessed.
In a recent meta-analysis and systematic review \textcites{cardinale_reliability_2017}
have identified over 75 studies that report psychometric results.
Albeit it's popularity the psychometric findings for the ICU are highly heterogeneous and not optimistic regarding the scales quality.
The present section will provide an overview of the most pervasive problematic fundings regularly reported in the assessment of the scales psychometric properties. 

\subsubsection{Dimensionality}
The ICU was constructed as unidimensional.
The scale originated from the extension of four APSD items from the CU traits subscale \parencite{kimonis_using_2015}.
ICU \#3  “I care about how well I do at school or work” (APSD \#3),
ICU \#5  “I feel bad or guilty when I do something wrong” (APSD \#12),
ICU \#6 “I do not show my emotions to others” (APSD \#19),
ICU \#8  “I am concerned about the feelings of others” (APSD \#18
As such one might argue the originally intended factor structure can actually assumed to be four dimensional.
Several studies have reported factorial solutions ranging from 2 to 4 factors. 
A three factor structure comprising the three factors unemotional, callous, and uncaringand the three factors unemotional, callous, and uncaring has first been suggested by \textcites{essau_callous-unemotional_2006}  
as the result of an exploratory factor analysis. 
This three factorial solution has been repeatedly replicated using CFA in international studies. It has furthermore been adjusted to a bi-factor model
with one overarching CU dimension  which is the presently most accepted factor structure.
(see \textcites{cardinale_reliability_2017}, as well as \textcites{kliem_factor_2019} for an overview).
Studies regularly report correlations of various behavioural outcomes with respect to these three subscales.
\textcites{kliem_factor_2019} have recently combined the idea of method factors and facets and suggested a four factor multi-trait multi method structure.
Hereby each item loads on a method factor indicated by its positive or negative wording, as well as on the content factor indicated by the original APSD item it was originally modeled after.

This solution has the advantage of also more closely corresponding to the four LPE specifiers.

Table \ref{tab:factor_solutions} provides an overview of the most commonly reported factor structures of the ICU.

\begin{ThreePartTable}
%	\begin{TableNotes}
%		\item \textit{Note.} 
%		\item .
%	\end{TableNotes}
	\begin{longtabu} to \linewidth {
			X[1,l]
			X[2,l]
			X[3,l]
			X[3,l]
			X[3,l]
			X[2,l]}
		\caption{\label{tab:factor_solutions}\protect\linebreak[1]
			\textit{Commonly reported ICU factor structures from the literature.}} \\
		\toprule
		\multicolumn{1}{l}{Factors} & \multicolumn{1}{l}{Reference} & \multicolumn{1}{l}{Uncaring} & \multicolumn{1}{l}{Unemotional}& \multicolumn{1}{l}{Callous} & \multicolumn{1}{l}{Careless}\\
		\midrule
		Factors & References & Uncaring & Unemotional & Callous & Careless \\
		3 & Essau et al 2006 & 3, 5, 13, 15, 16, 17, 23, 24 & 1, 6, 14, 19, 22 & 2, 4, 7, 8, 9, 10, 11, 12, 18, 20, 21 &  \\
		3 & Kimonis et al 2015 & 3, 5, 13, 15, 16, 17, 23, 24 & 1, 6, 14, 19, 22 & 4, 7, 8, 9, 11, 12, 18, 20, 21 &  \\
		4 & APSD model Kliem et al 2019 & 4, 8, 12, 17, 21, 24 & 1, 6, 10, 14, 19, 22 & 2, 5, 9, 13, 16, 18 & 3, 7, 11, 15, 20, 23 \\
				\midrule
		&  & Negative &  & Positive &  \\
				\midrule
		2 & Method factors & 1, 3, 5, 8, 13, 14, 15, 16, 17, 19, 23, 24 &  & 2, 4, 6, 7, 9, 10, 11, 12, 18, 20, 21, 22 & \\
		\bottomrule
%\insertTableNotes 
\end{longtabu}
\end{ThreePartTable}

\subsubsection{Validity}
The Unemotional dimension capturing shallow affect has been repeatedly criticized (Hawes, Byrd, et al., 2014;
Kimonis, Bagner, Linares, Blake, \& Rodriguez, 2014) \todo{add proper references}

It consistently shows a lower internal consistency and low to null correlations with criterion measures \parencite{cardinale_reliability_2017}.

Explanations for this lack or predictive validity is frequently attributed to the item wording which addresses restricted emotional expression. 
Cardinale and March thus suggest the items from the unemotional subscale might tap into internalizing disorder or anhedonia  \parencite*{cardinale_reliability_2017}.
In a similar vein, Viding and Kimonis hypothesize that the ICU emotionality items might capture behavior that can also be observed in other contexts such as autism or depression \parencite*{viding_callousunemotional_2018}. 
Cardinale and Marsh suggest the inclusion of emotions that are more closely tailored to the affective deficits observed in CU traits while allowing emotions such as anger happiness or disgust that have not been showed to be related to psychopathy and callousness (Blair, 2012; S. W. Hawes et al., 2014; Hicks \& Patrick, 
2006; Kimonis et al., 2012; Urben et al., 2016) Dawel, O’Kearney, McKone, \& Palermo, 
2012; Jones, Happé, Gilbert, Burnett, \& Viding, 2010; 
Marsh \& Blair, 2008; Marsh et al., 2011) \todo{add proper references} 
This is in line with the recommendations of Viding and Kimonis who suggest a revision for the unemotional items to specifically
refer to lack of emotional response in an interpersonal context such as not being moved by someones joy of sorrow.
This would furthermore correspond more closely to the LPE specifier who terms this dimension "shallow or deficient affect" (\nptextcite{DSM-5}, p. 471)
and define it as  not expressing feelings or showing emotions except in a shallow, insincere or superficial way. 
The LPE specifier furthermore states that emotions can indeed be shown and turned on and off quickly if used for personal gain (e.g. intimidation or manipulation). 
Another aspect is entirely missing from the unemotional items contained in the ICU.


\subsubsection{Problematic Items}
In addition to the commonly reported problems with the factor structure as well as the unemotional subscale there are a number of individual items that have been criticized for their
insufficient psychometric performance.

Item xy \todo{add reference and item number}
has reputedly been criticized for its low association with the rest of the scale. ...

\subsubsection{German Translation}
The ICU has been translated to German by \todo{add information about translation}


\subsubsection{Role of the ICU in the Revision of the DSM}
http://labs.uno.edu/developmental-psychopathology/ICU.html

ICU items were used in the secondary data analyses that guided the DSM-5 specifier's formation
Frick \& Moffitt, 2010 \todo{figure out what frick and moffitt 2010 refers to}

Lack of Remorse or guilt
Callous lack of empathy
Shallow or deficient affect
Unconcerned about performance / lack of concern about performance

Original Model of the ASPD (map to the LPE criteria):
Uncaring (4, 8, 12, 17, 21, 24),
Unemotional (1, 6, 10, 14, 19, 22),
Callous (2, 5, 9, 13, 16, 18), and
Careless (3, 7, 11, 15, 20, 23)


Lahey 2014 demands that :
" In my view, priority should be given to future research on the CU construct,
the items that define it,
and the psychological processes that these putative dispositional definitions reflect. (p.3)"
the correlation of callousness with fear and anxiety also deserves study

- Research Domain Criteria movement genetic variants, maladaptive experiences might line up better with specific psychological processes
- it is essential to determine if CU is related to important external criterion variables over and above measures of the severity of CD
- : 
"Thus we suggest that future research should not be limited to studies using DSM-5 criteria for either CD or the specifier "with limited Prosocial Emotions.""
- Some items (or combination of items might have higher predictive validity than traits that consist of similar items) combination of different items might encompass more outcome-relevant content than combinations of similar items


\subsection{Psychological Network Analysis}

The following section provides an introduction into the data analytical method applied in the present study.

\subsection{Common cause theory}
Latent variable models (such as factor analytical models in structural equation modeling (SEM)) model data that is theorized to stem from a common cause.
Here items are viewed as passive, interchangeable indicators of a underlying construct.
Interventions would focus on the latent variable.
The covariance of phenomena is explained by an underlying latent factor (i.e. a disorder).
The existence of a latent factor is a fundamental assumption in the common cause framework.
A second underlying assumption of factor analytical models is local independence. 
Local independence refers to the idea that phenomena (i.e. items or symptoms) are not directly related.
Their covariance is expected to disappear after taking the common latent factor into account.
This makes immediate sense for symptoms of a medical condition (e.g. fewer and a rash) to be expression of an underlying infection and to be unrelated after taking the infection into account. 
However, for the emergence of psychological conditions complex interactions and reinforcements of symptoms are often believed to be characteristic for the emergence as well as the continuance of some disorders.
A third assumption of the common cause model is th that correlations between scales or syndromes result from a direct relationship between the underlying latent variables. An alternative explanation might be immediate interactions at the level of behavioral items and symptoms.   

\subsection{Network theory}
This view of items as important autonomous causal agents is central to network theory.
Items are not merely seen as symptoms but as elements that should be studied and that interventions could target.
A disorder or construct might not exist as an entity but might be best described as emerging property from the interactions of the items in the system. 

%
%Such interactions/ interplay or reinforcing cycles are also imaginable for the items constituting CU-traits.
%Not caring about others feelings might result in anti-social behavior. The lack of care consequently causes a lack of remorse for bad actions which might again cause a lack of apology.
%Observing ones own behavior of not apologizing as well as lack of emotional response to the violation of social norms might again reinforce the implicit understanding of not caring.


\subsection{Psychological Network analysis}
Psychological networks assess the conditional dependence relationships of symptom or item sets.
The symptoms/items are represented as nodes in a graph. 
Their association is represented by an edge connecting the nodes.  
An edge in the network hence illustrates how two items (or symptoms) are related after controlling for all other items in the network. The strength of this association (i.e. the edge weight) is commonly visualized by the width and saturation of the edge.

Network analysis models the unique variances among each pair of items (instead of the shared variances in SEM) typically in the form of node wise regression or partial correlations. 

If an edge between two nodes is present this means that the two nodes are not independent (conditional on all other nodes).

%Depending on the underlying node distribution...
%Normally distributed nodes = \gls{ggm} Gaussian graphical model (GGM)
%These edge weights can be estimated via node wise regression or more efficiently as partial correlation coefficients
%Partial correlations can be obtained from the correlation matrix, hence the raw data is not needed.
% \gls{mgm} Mixed Graphical Models (MGM)
Network analysis can hence help to identify the core characteristics of a psychological construct. 
A central well connected node likely represents a symptom that is very important.
Inspection of a network graph can furthermore show substructures in the construct for example nodes with a very high interconnectivity or nodes that are quite marginal and not related to other nodes in the network.

Except for the visual interpretation of the resulting networks the network approach comes with en entire toolbox.
Several node and network indices can be interpreted.
These indices might reflect properties of the whole network  such as how dense it is
(i.e. how many of the nodes are non-zero) or whether patterns of node clustering (i.e. communities) can be observed.

All of these indices can then be related to potential properties and relationships of corresponding items.

\subsubsection{Network Psychometrics}
Psychometric properties of psychological scaled are traditionally assessed within a structural equation modeling (SEM) framework.
As pointed out in the previous section, network analysis and SEM have different underlying assumptions. 
Nevertheless, they have shown to be conceptually related \parencite{epskamp_network_2018}.
Hence several network properties typically assessed in the context of a psychological network analysis
be interpreted as psychometric properties similar to those obtained from factor analysis. 
The recently emerged field of assessing psychometric properties via network analysis has been coined \textit{network psychometrics} \parencite{epskamp_network_2018}.

Node centrality is a measure frequently assessed in the psychometric network literature. 
It assesses the (absolute or relative) sum of all edges connected to a specific node.
Recent studies have shown, that note centrality is related to factor loadings \parencite{mottus_why_2018}.

Similarly, dimensionality, an important aspect in psychometric analyses in the SEM framework, is also an important property of psychological networks.
The main method to assess dimensionality in psychological networks is via community detection algorithms.
While confirmatory approaches are currently being developed the traditional approaches are inherently exploratory and akin to exploratory factor analysis.
Several algorithms are currently being discussed and evaluated in the current network psychometrics literature \parencite{christensen_towards_2020}.
The walktrap algorithm has emerged as one of the most popular and especially suited for small, dense, correlation based networks \parencite{gates_monte_2016}. 
The walktrap algorithm identifies the optimal partitioning of nodes by assessing modularity. 
Modularity measures the strength of in group connections for a given node set compared to what would be expected by random partitioning.
Dimensionality detection by estimation of a partial correlation network in combination with the walktrap algorithm has been found to be 
similar yet superior to traditional methods of exploratory factor analysis and suitable for dimensionality detection of psychological scales \parencite{golino_exploratory_2017}.

\subsubsection{Parallels of Psychological network analysis and Structural equation modeling}
The above mentioned comparability of certain network analysis indices counterpart is not by chance.
Network models are equivalent to latent variable models under a set of conditions.
There are equivalent latent variable models for every network model and vice versa.
Specifically, data generated from a one-factorial model for example will result in a fully connected network with one community. 
Golino and Epskamp have referred to this as the fundamental rule of network psychometrics: "Clusters in network = latent variables" (\nptextcite, p.4).

Despite these commonalities it has to be kept in mind that the underlying data generating perspective is fundamentally different. 
These different perspectives are also reflected in the way items are constructed. 
Personality inventories are often designed from a latent variable perspective.
Item redundancy, for example, is a feature enabling a latent factor to be assessed with higher precision,
from a network analysis perspective however, where each node is an causal agent in its own right, redundancy is a nuisance. 


\subsubsection{Present Applications of Network Theory to CU Traits and Psychopathy}
Despite its relatively short history, network theory has already been applied in a number of studies in the field of psychopathy.
\textcites{verschuere_what_2018} found callousness/lack of empathy item to be the most central item in two large U.S. offender samples of 
\textcite{bronchain_network_2019}

\subsection{Multi-verse Analysis}

Partial correlations can be easily obtained by inverting and standardizing the sample covariance matrix. 
Unfortunately, in practice the estimation is not as straight forward as that.
Psychological data is typically characterized by unfavorable properties such as violations of multivariate normality and
small sample sizes.

Small sample sizes typically lead to unstable parameter estimates due to overfitting whereas distributional violations might make the use of zero-order correlations inappropriate.

Several techniques have recently been suggested and implemented to overcome these issues. 

Overfitting is typically combated by the application of shrinkage. 
Large overoptimistic parameter estimates as well as small spurious associations are prevented which leads to conservative and more stable parameter estimates as well as a sparse solution. 
Distributional violations are tackled via variable transformation as well as the use of alternative correlation methods such as polychoric correlations \parencite{epskamp_tutorial_2018}. 

Network analysis is a fairly recent tool in the psychological sciences.
Hence, methodological developments are rapidly progressing.
The stability of network analyses and hence the reproducibility and generalizability of results is currently debated. \todo{add reference of stability criticism}
To address both methodological debate as well as stability concerns the present study will apply a multi-verse approach.
The multi-verse approach acknowledges that there are a multitude of decisions that regard the exact implementation of the data analytical process.
Instead of (more or less)arbitrarily settling on one possible implementation, a meta-verse analysis implements all reasonable analyses and displays a multitude of obtainable results instead of a single point estimate.


\subsection{Research Questions}
It is commonly accepted that common methodological routines greatly influence the way we design measurement tools and design studies.
A common example it the optimization of Chronbach's alpha which originally intended to improve reliability might in extreme cases merely results in asking the same question multiple times.
It is hence beneficial to occasionally apply new techniques to old problems to see whether the conclusion align or different insight might be gained.

In this vein this thesis comprises a psychometric network analysis of the ICU to investigate whether the psychometric issues reported in the literature can be reproduced with this
alternative approach.

Insights dimensionality
item coding four subscales or three
Further insights in the role of problematic items

As the psychometric shortcomings have been hypothesized to potential issues with the LPE specifier \parencite{cardinale_reliability_2017} 

The present analysis has been preregistered on the Open Science Framework \todo{add link}. 
Deviations from the preregistration are elaborated on in appendix \todo{add reference to appendix} 



